\input{$HOME/studyproject/universe/history/preamble_beamer_rus.tex} % here is document's settings for russian
%\input{$HOME/studyproject/universe/history/preamble_beamer_eng.tex} % here is document's settings for english


\title{Крест за взятие Измаила}
\author{Немков Н.М.}
\institute[МГТУ]{МГТУ им. Н.Э. Баумана}
\date{25.09.2023}
\logo{\includegraphics[width=1cm]{logo}}

\begin{document}

\begin{frame}
\maketitle
\end{frame}

\section{}

\begin{frame}{Появление награды}

	11 декабря 1790 года под командыванием А.В. Суворова, армия взяла крепость Измаил. В честь этого 25 марта 1791 года, императрицей Екатериной II, была учереждена государственная награда - \textbf{Крест "За взятие Измаила"}

\end{frame}
\begin{frame}{}
	Не смотря на учреждение награды 25 марта 1791 года чеканка крестов была отложена из-за смерти главнокомандующего светлейшего князя Г. А. Потёмкина и начата лишь в марте 1792 года. Вручение награды офицерам продолжалось вплоть до начала царствования императора Александра I.
\end{frame}
\begin{frame}{Внешнее описание}
	\begin{columns}
		\begin{column}{0.45\textwidth}
			\includegraphics[width=0.9\textwidth]{$HOME/studyproject/universe/history/medal.jpg}
		\end{column}
		\begin{column}{0.45\textwidth}

			Крест был сделан из золота. Размер креста — 46 × 46 мм. Крест равносторонний четырёхконечный с закруглёнными окончаниями. С двух сторон нанесены надписи.

		\end{column}
	\end{columns}
\end{frame}
\begin{frame}{}
	\begin{columns}
		\begin{column}{0.45\textwidth}

			Крест имел ушко для крепления к ленте. Носили его на груди на Георгиевской ленте.

		\end{column}
		\begin{column}{0.45\textwidth}

			\includegraphics[width=0.9\textwidth]{$HOME/studyproject/universe/history/medal2.jpg}

		\end{column}
	\end{columns}
\end{frame}
\begin{frame}{}
	\begin{columns}
		\begin{column}{0.45\textwidth}
			\includegraphics[width=0.9\textwidth]{$HOME/studyproject/universe/history/shlem3.jpg}
		\end{column}
		\begin{column}{0.45\textwidth}

		\end{column}
	\end{columns}
\end{frame}

\section{References}

\begin{frame}[t]{References}
	\printbibliography
	\url{}
	\url{}
\end{frame}

\section{Благоданость}
\begin{frame}
	\centering
	\huge
	Спасибо за внимание!
\end{frame}

\end{document}
