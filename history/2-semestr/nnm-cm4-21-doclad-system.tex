\documentclass{beamer}

\usepackage{cmap}
\usepackage[T1,T2A]{fontenc}        % add eng,rus encoding support
\usepackage[utf8]{inputenc}         % add UTF8 support
\usepackage[english,russian]{babel} % add eng,rus(base) package

% Use it for English document
%\usepackage[utf8]{inputenc} % add UTF8 support
%\usepackage{fontspec}       % to use any font known to the operating system
%\setmainfont{PT Serif}      % set defolt font

\usepackage{amsmath, amsfonts, amssymb, amsthm, mathtools} % add math support

\linespread{1}               % length between str
\setlength{\parindent}{16pt} % red str
\setlength{\parskip}{6pt}   % length between paragraphs

\usepackage[backend=biber, style=authoryear-icomp]{biblatex}
\addbibresource{$HOME/latex-templates/biblio.bib}            % path to bibliography base

\usetheme{Madrid}
\setbeamertemplate{frametitle}[default][center]

\renewcommand{\thefootnote}{\arabic{footnote}}
 % here is document's settings for russian
%\input{$HOME/studyproject/universe/history/preamble-beamer-eng.tex} % here is document's settings for english


\title{Система Образования}
\author{Немков Н.М.}
\institute[МГТУ]{МГТУ им. Н.Э. Баумана}
\date{14.02.2024}
\logo{\includegraphics[width=1cm]{images/logo}}

\begin{document}

\begin{frame}
\maketitle
\end{frame}

\begin{frame}{Содержание}
\tableofcontents
\end{frame}
\section{Состояние образования в России до реформы}
\begin{frame}{Состояние образования в России до реформы}

\Large{
	Система образования была достаточно устаревшей и не эффектифной, а учебные заведения были доступны только для ограниченного числа людей

	Образовательная система сильно отстовала от западноевапейской, также учебники были давно устаревшими
}

\end{frame}

\section{Реформа министерства нородного просвещения}
\begin{frame}{Реформа министерства нородного просвещения}

	Основные задачи министерства:
	\begin{itemize}

	\item координация деятельности образовательных учреждений

	\item разработка учебных программ,

	\item подготовка учителей,

	\item финансирование образования,

	\item надзор за качеством образования и

	\item управлением образовательными учереждениями на всей территории России
	\end{itemize}
\end{frame}

\begin{frame}
	\Large{
	Позволило создать единую систему образования

	Значительно увеличилось колличество учеников

	Улучшилось качество преподования}
\end{frame}


\section{Расширение сети народных училищ и университетов}
\begin{frame}{Расширение сети народных училищ и университетов}

	\Large{В результате реформы колличество народных училищ значительно возрасло, а их программа стала более разнообразной и качественной}

\end{frame}

\begin{frame}{}

	\Large{Были созданны новые университеты и приглашенны высококвалифицированные учителя и профессора из Европы}

\end{frame}
\section{Реорганизация системы среднего образования}
\begin{frame}{Реорганизация системы среднего образования}

	\Large{В школьную программу были включены новые предметы

	Были разработанны учебники и методические материалы для учителей

	Были введены государственне экзамены}

\end{frame}

\section{Отмена ограничений по поступлению в университет крестьян и иностранцев}
\begin{frame}{Отмена ограничений по поступлению в университет крестьян и иностранцев}
	Крестьяне были исключены из числа абитуриентов поскольку не могли получить свидетельство о собственности на землю

	1861г. -- была проведена крестьянская реформа

	Иностранцы могли поступить в университет только при наличии сппециального разрешения правительства

	1863г. -- было принято решение отменить данное ограничение
\end{frame}

\section{Последствия реформы}
\begin{frame}{Последствия реформы}
	\begin{itemize}
		\item Расширение сети учебных заведений
		\item Улучшение качества образования
		\item Стимулирование научных исследований
		\item Развитие культуры и литературы
		\item Рост числа студентов
		\item усилеине политических разногласий
	\end{itemize}
\end{frame}

\section{Источники}

\begin{frame}[t]{Источники}
	\Large{
	\url{https://rusistori.ru/rossiyskaya-imperiya/reforma-obrazovaniya-pri-aleksandre-ii-provedennaya-v-1864-godu/}

	\url{https://historykratko.ru/19-vek/reforma-obrazovaniya-aleksandra-2/}}
\end{frame}


\section{Благоданость}
\begin{frame}
	\centering
	\huge
	Спасибо за внимание!
\end{frame}

\end{document}
