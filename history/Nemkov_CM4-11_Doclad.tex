\input{$HOME/studyproject/universe/history/preamble_beamer_rus.tex} % here is document's settings for russian
%\input{$HOME/studyproject/universe/history/preamble_beamer_eng.tex} % here is document's settings for english

\title{Шлем Ярослава Всеволода}
\author{nils-nm}
\institute[МГТУ]{МГТУ им. Н.Э. Баумана}
\date{25.09.2023}
\logo{\includegraphics[width=1cm]{logo}}

\begin{document}

\begin{frame}
\maketitle
\end{frame}

\section{Шлем Ярослава Всеволода}

\begin{frame}{История шлема}

	Будучи смертельно больным, отец передал Ярославу Переяславль-Залесский. В конфликте, возникшем после смерти отца между старшими братьями, Константином и Юрием, Ярослав поддержал Юрия и был разбит вместе с ним в Липицкой битве 1216 года. Во время битвы он потерял свой шлем, который был найден в 1808 году.

\end{frame}
\begin{frame}{Состояние шлема}
	\begin{columns}
		\begin{column}{0.45\textwidth}
			\includegraphics[width=0.9\textwidth]{$HOME/studyproject/universe/history/shlem.jpg}
		\end{column}
		\begin{column}{0.45\textwidth}

			Тулья\footnote{Основная, верхняя часть головного убора (без полей, околыша, козырька и т.п.).} шлема сохранилась плохо — в виде двух крупных частей, поэтому нельзя точно определить её форму и конструкцию.

			Корпус шлема покрыт серебряным листом, украшен позолоченными серебряными чеканными накладками.

		\end{column}
	\end{columns}
\end{frame}
\begin{frame}{Состояние шлема}

	Наверху — образа Вседержителя, святых Георгия, Василия и Феодора. На налобной пластине — образ Архангела Михаила и черневая надпись: «Вьликъи архистратиже ги Михаиле помози рабу свуему Феодору»\footnote{Великий Архангел Михаил, помоги слуге Твоему Феодору}. По краю шлема — орнаментальная кайма.

\end{frame}
\begin{frame}{Состояние шлема}
	\begin{columns}
		\begin{column}{0.45\textwidth}

	По данным Бориса Колчина, тулья шлема является цельнокованной и изготовлена из железа или малоуглеродистой стали техникой штамповки с последующей выколоткой, что отличает его от других шлемов данного типа, равно как и других типов данного периода. При изготовлении шлема она была предварительно набита серебряным листом.

		\end{column}
		\begin{column}{0.45\textwidth}

			\includegraphics[width=0.9\textwidth]{$HOME/studyproject/universe/history/shlem2.jpg}

		\end{column}
	\end{columns}
\end{frame}
\begin{frame}{Состояние шлема}
	\begin{columns}
		\begin{column}{0.45\textwidth}
			\includegraphics[width=0.9\textwidth]{$HOME/studyproject/universe/history/shlem3.jpg}
		\end{column}
		\begin{column}{0.45\textwidth}

			К 2021 году шлем был впервые отреставрирован реставратором музеев Московского Кремля Михаилом Кружалиным.

		\end{column}
	\end{columns}
\end{frame}

\section{References}

\begin{frame}[t]{References}
	\printbibliography
	\url{https://dic.academic.ru/dic.nsf/ushakov/1060158}
	\url{https://stuki-druki.com/authors/Yaroslav-Wsevolodovich.php}
\end{frame}

\section{Благоданость}
\begin{frame}
	\centering
	\huge
	Спасибо за внимание!
\end{frame}

\end{document}
