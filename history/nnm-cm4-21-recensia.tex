% template - article
\input{$HOME/latex-templates/preamble_article_rus.tex} % here is document's settings for russian
%\input{$HOME/latex-templates/preamble_article_eng.tex} % here is document's settings for english

\title{\so{MEIN REZEPTBUCH}}
\author{max-wn}
\date{April 12, 1961}

\begin{document}

\maketitle
\newpage

\large{Полёт первого человека в космическое пространство стал одним из знаковых событий ХХ века. Этот успех заметно упрочнил позиции Советского Союза на международной арене. Однако на сегодняшний день освещение истории формирования первого отряда космонафтов явно устарело. Больший интерес представляет то, как формировалась и по каким принципам работала система отбора в космонавты. Именно об этом пишет В.С. Батченко в своей статье "Первый набор в космонавты: от идеии к воплащению".}


В данной работе опысывается иследования на тему отбора людей для полета в космос. Началось все с секретного постановления Совета министров СССР 30 декабря 1949 года, "О дальнейшем развитии работ по исследованию верхних слоев атмосферы". Согласно ему в 1950-1951 году предусматривались подготовка и проведение серии запусков первой советской ракеты Р-1, сопроваждемых геофизическими и медико-биологическими исследованиями. Группа военных врачей Института авиамедицины занялась изучением влияния космических факторов на животных, в первую очередь собак. После успешног ополёта собаки Лайки (ноябрь 1957 год) для решения каким должен быть первый полет человека в космос, в ОКБ-1 создали две группы. одна должна было проработать баллистический полёт другая орбитальный.


\end{document}
